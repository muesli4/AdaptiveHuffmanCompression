\documentclass[utf8x]{beamer}

\usepackage[utf8x]{inputenc}
\usepackage{ucs}
\usepackage{lmodern}
\usepackage[ngerman]{babel}
\usepackage{enumerate}
\usepackage{listings}
\usepackage{hyperref}

\usetheme{Madrid}

\subject{Datenkompression}

\title{Adaptive Huffman Komprimierung}

\date{16. Mai 2013}

\author{Niklas Jugl, Moritz Bruder}


%\setbeamercovered{transparent}




\setbeamertemplate{footline}
{
  \leavevmode%
  \hbox{%
  \begin{beamercolorbox}[wd=.333333\paperwidth,ht=2.25ex,dp=1ex,center]{author in head/foot}%
    \usebeamerfont{author in head/foot}\insertshortauthor%~~\beamer@ifempty{\insertshortinstitute}{}{(\insertshortinstitute)}
  \end{beamercolorbox}%
  \begin{beamercolorbox}[wd=.333333\paperwidth,ht=2.25ex,dp=1ex,center]{title in head/foot}%
    \usebeamerfont{title in head/foot}\insertshorttitle
  \end{beamercolorbox}%
  \begin{beamercolorbox}[wd=.333333\paperwidth,ht=2.25ex,dp=1ex,right]{date in head/foot}%
    \usebeamerfont{date in head/foot}\insertshortdate{}\hspace*{2em}
    \insertframenumber{} / \inserttotalframenumber\hspace*{2ex} 
  \end{beamercolorbox}}%
  \vskip0pt%
}

\setbeamertemplate{headline}
{%
  \leavevmode%
  \begin{beamercolorbox}[wd=.5\paperwidth,ht=2.5ex,dp=1.125ex]{section in head/foot}%
    \hbox to .5\paperwidth{\hfil\insertsectionhead\hfil}
  \end{beamercolorbox}%
  \begin{beamercolorbox}[wd=.5\paperwidth,ht=2.5ex,dp=1.125ex]{subsection in head/foot}%
    \hbox to .5\paperwidth{\hfil\insertsubsectionhead\hfil}
  \end{beamercolorbox}%
}


\begin{document}

    \defverbatim\equalityCode{%
        \begin{lstlisting}
nodes[node.number] == node
        \end{lstlisting}
    }

    \frame{

        \titlepage
    }

    \frame{
	    \frametitle{Inhaltsverzeichnis}
	    \tableofcontents
    }

    %   - array als hilfsdatenstruktur für BFS
    %       - iteration über die knoten im baum zuerst in die breite (da update die baumstruktur aufrecht erhält)
    %       - da häufige zeichen oben im baum sind ist suche nach symbol im mittel effizient (zumindest wenn die häufigkeiten sich gut unterscheiden)
    %           WICHTIG: kein (binärer) suchbaum
    %       - für jeden knoten node gilt: nodes[node.number] == node

    
    \section{Array als Hilfsdatenstruktur für Breitensuche}
    \frame{\tableofcontents[currentsection]}
    \begin{frame}[<+->]
    
        \begin{block}{Überlegungen}
            \begin{itemize}
                \item
                    Es handelt sich nicht um einen (binären) Suchbaum
            
                \item
                    Suche von Knoten über die Breite ist effizienter, da häufige Knoten weiter oben sind
                
                \item
                    Knotennummern steigen nach oben und rechts an
                    
            \end{itemize}
        \end{block}
    \end{frame}
    \begin{frame}[<+->]

        \begin{block}{Umsetzung}
            \begin{itemize}
                \item
                    Deshalb gilt für jeden Knoten:
                    \begin{quote}
                        \equalityCode
                    \end{quote}
                \item
                    Array wächst von oben herunter
            \end{itemize}
        \end{block}
    \end{frame}


    %   - bit ausgabe und eingabe
    %       - buffere in einem byte und gebe dies dann an byte basierte streams weiter
    %
    
    \section{Bitweise Ein- und Ausgabe}
    \frame{\tableofcontents[currentsection]}
    \begin{frame}[<+->]
    
        \begin{block}{Problem}
            Addressierung findet auf Byteebene statt.

        \end{block}
        
        \begin{block}{Lösung}
            \begin{itemize}
                \item
                    1 Byte zwischenspeichern

                \item
                    Manipuliere mit Bitoperatoren zum Lesen und Schreiben

            \end{itemize}
        \end{block}
    \end{frame}
    %   - dekodieren
    %       - baum absteigen
    %           - links wenn 0 gelesen
    %           - rechts sonst
    %       - knoten = symbol? symbol ausgeben
    %       - sonst ist knoten nyt, repräsentation für symbol empfangen
    %
    \section{Dekodieren}
    \frame{\tableofcontents[currentsection]}
    \begin{frame}[<+->]
    
        \begin{enumerate}[1.]
            \item
                Baum absteigen (links wenn 0 gelesen, rechts sonst)

            \item
                Knoten mit Symbol gefunden? Gebe Symbol aus
            \item
                Ansonsten empfange Symbol

        \end{enumerate}

    \end{frame}
    %   - enkodieren
    %       - suche im array nach zeichen
    %           - wenn nicht gefunden dann nyt, hänge kodierung umgekehrt an
    %           - wandere den baum hoch, prüfe ob vater diesen knoten als linkes kind hat
    %               - hänge 0 an kodierung
    %               - 1 sonst
    %           - drehe die folge um
    
    \section{Enkodieren}
    \frame{\tableofcontents[currentsection]}
    \begin{frame}[<+->]
    
        \begin{enumerate}[1.]
            \item
                Lineare Suche nach Knoten mit Symbol im Array

            \item
                Wenn nicht gefunden, dann handelt es sich um NYT, hänge Symbol umgekehrt an
            \item
                Wandere den Baum hoch, wenn momentaner Knoten linkes Kind war hänge 0 an, sonst 1
            \item
                Drehe Kodierung um

        \end{enumerate}

    \end{frame}
    %   - update
    %       - nyt
    %           - hänge nyt an neuen knoten links und neuen knoten für dekodiertes symbol rechts
    %
    %       - block konzept
    %       - wandere den baum hoch, wenn knoten nicht geschwister von nyt und nicht maximale nummber im block hat
    %           - tausche mit maximaler nummer im block (knoten steigt quasi in die nächste "gewichtsklasse" auf)
    %           - erhöhe dann gewicht


    \section{Baum aktualisieren}
    \frame{\tableofcontents[currentsection]}
    \begin{frame}[<+->]
        Bereits in der Vorlesung besprochen.
        \pause
        \begin{block}{Blockprinzip}
            Ein Block besteht aus allen Knoten mit gleichem Gewicht.
        \end{block}
        \pause
        \begin{enumerate}[1.]
            \item
                Wird Knoten zu Symbol nicht gefunden hänge einen neuen ein
            \item
                Wandere den Baum nach oben und wenn der aktuelle Knoten nicht der mit der größten Nummer im Block ist, tausche diesen mit dem größten aus
            \item
                Erhöhe dann das Gewicht
        \end{enumerate}
    \end{frame}

    \section{Sonstiges}
    \frame{\tableofcontents[currentsection]}
    \begin{frame}[<+->]
    
        \begin{itemize}
            \item
                Binaries

            \item
                Zu finden auf \href{https://github.com/muesli4/AdaptiveHuffmanCompression/tree/ugly}{github.com} unter \emph{AdaptiveHuffmanCompression}
                
            \item
                Calgary corpus

        \end{itemize}

    \end{frame}
    %   - binaries und git projekt
    %       - Encoder
    %       - Decoder
    %       - Main
    %       - Suche nach AdaptiveHuffmanCompression (momentan noch im ugly branch)
    %
    %   - ergebnis mit calgary corpus
    %
    
\end{document}
